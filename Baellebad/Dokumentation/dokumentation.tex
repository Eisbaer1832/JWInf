\documentclass[a4paper,10pt,ngerman]{scrartcl}
\usepackage{babel}
\usepackage[T1]{fontenc}
\usepackage[utf8x]{inputenc}
\usepackage[a4paper,margin=2.5cm,footskip=0.5cm]{geometry}

% Die nächsten vier Felder bitte anpassen:
\newcommand{\Aufgabe}{Aufgabe 1: Bällebad} % Aufgabennummer und Aufgabennamen angeben
\newcommand{\TeamId}{00015}                       % Team-ID aus dem PMS angeben
\newcommand{\TeamName}{TGG-Abi26}                 % Team-Namen angeben
\newcommand{\Namen}{Jannes Lehmann, Tino Brinker}           % Namen der Bearbeiter/-innen dieser Aufgabe angeben
\usepackage{float}
% Kopf- und Fußzeilen
\usepackage{scrlayer-scrpage, lastpage}
\setkomafont{pageheadfoot}{\large\textrm}
\lohead{\Aufgabe}
\rohead{Team-ID: \TeamId}
\cfoot*{\thepage{}/\pageref{LastPage}}

% Position des Titels
\usepackage{titling}
\setlength{\droptitle}{-1.0cm}

% Für mathematische Befehle und Symbole
\usepackage{amsmath}
\usepackage{amssymb}

% Für Bilder
\usepackage{graphicx}

% Für Algorithmen
\usepackage{algpseudocode}

% Für Quelltext
\usepackage{listings}
\usepackage{color}
\definecolor{mygreen}{rgb}{0,0.6,0}
\definecolor{mygray}{rgb}{0.5,0.5,0.5}
\definecolor{mymauve}{rgb}{0.58,0,0.82}
\lstset{
	keywordstyle=\color{blue},commentstyle=\color{mygreen},
	stringstyle=\color{mymauve},rulecolor=\color{black},
	basicstyle=\footnotesize\ttfamily,numberstyle=\tiny\color{mygray},
	captionpos=b, % sets the caption-position to bottom
	keepspaces=true, % keeps spaces in text
	numbers=left, numbersep=5pt, showspaces=false,showstringspaces=true,
	showtabs=false, stepnumber=2, tabsize=2, title=\lstname
}


% Diese beiden Pakete müssen zuletzt geladen werden
%\usepackage{hyperref} % Anklickbare Links im Dokument
\usepackage{cleveref}

% Daten für die Titelseite
\title{\textbf{\Huge\Aufgabe}}
\author{\LARGE Team-ID: \LARGE \TeamId \\\\
\LARGE Team-Name: \LARGE \TeamName \\\\
\LARGE Bearbeiter/-innen dieser Aufgabe: \\
\LARGE \Namen\\\\}
\date{\LARGE\today}

\begin{document}

	\maketitle
	\tableofcontents

	\vspace{0.5cm}
	\newpage

	\section{Lösungsidee}
		Für jeden Wochentag werden jeweils die relevanten Schulstunden und ihre benötigten Bälle gezählt.
		Die Stunde mit der höchsten benötigten Ballzahl spiegelt somit die Gesamtzahl der benötigten Bälle wieder.

	\section{Umsetzung}
		Die Lösung dieser Aufgabe wurde in zwei separate Modulen interpretiert.

		\begin{enumerate}
			\item Baellebad.py zur Analyse der Daten und Berechnung des Ergebnisses
			\item GUI.py zur grafischen Darstellung des Ergebnisses.
		\end{enumerate}


		Zuerst werden die Daten eingelesen und jeweils in einer Liste vom Typ Unterricht gespeichert.
		Daraufhin wird durch die Liste der Unterrichte iteriert und die Anzahl der Bälle auf die vorhandenen Bälle addiert.
		\begin{lstlisting}[language=Python, caption={Berechnung des Ballbedarfs pro Stunde},label={lst:code1}]
		for item in classes:
			hours = getHours(item.Start, item.End) # Gibt eine Liste mit den relevanten Stunden zurück
			day_time_stamps = days[item.Day]
			for hour in hours:
				if isSport(item.Class):
					lastVal = day_time_stamps.get(hour)
					if lastVal is None: lastVal = 0
					# erhöht die Anzahl der für die Stunde benötigten Bälle
					day_time_stamps[hour] = lastVal + item.BallsNeeded
			days[item.Day] = day_time_stamps
		\end{lstlisting}

		Um zu ermitteln in welchen Stunden ein Ball benötigt wird, wird für jede Stunde des Bereich [start;end] ein Element in die Liste "hours" hinzugefügt.
		\begin{lstlisting}[language=Python, caption={Berechnung der relevanten Stunden},label={lst:code2}]
			def getHours(start, end):
				c = start
				hours = []
				while c <= end:
					hours.append(c)
					c = c + 1
				return hours
		\end{lstlisting}


		Die Werte werden in einem nach Tagen sortiertem Dictionary, in welchem die Liste der Stunden gespeichert wird gespeichert.
	 	\lstinline|{"Montag": {[8, 9, ...], ...}|

		Zuletzt wird durch die gespeicherten Werte iteriert und der maximale Bedarf berechnet.

		Wird der Algorithmus durch das GUI ausgeführt, werden die errechneten Werte mithilfe von PyQt6 und pyqtgraph dargestellt.


	\section{Werkzeuge}
		Für das GUI wurden folgende drittanbieter Libraries verwendet:
		\begin{enumerate}
			\item PyQt6
			\item pyqtgraph
		\end{enumerate}
		Es wurden keine weiteren Hilfsmittel verwendet.

	\newpage
	\section{Beispiele}

	\subsection{Beispiel 00}
		\begin{figure}[H]
			\centering
			\includegraphics[scale=0.65]{exports/ball00.png}
		\end{figure}
		Die Schule braucht höchstens 15 Bälle am Montag um 15 Uhr.

	\subsection{Beispiel 01}
		\begin{figure}[H]
			\centering
			\includegraphics[scale=0.65]{exports/ball01.png}
		\end{figure}
		Die Schule braucht höchstens 65 Bälle am Mittwoch um 11 Uhr.
	\subsection{Beispiel 02}
		\begin{figure}[H]
			\centering
			\includegraphics[scale=0.65]{exports/ball02.png}
		\end{figure}
		Die Schule braucht höchstens 174 Bälle am Montag um 11 Uhr.
	\subsection{Beispiel 03}
		\begin{figure}[H]
			\centering
			\includegraphics[scale=0.65]{exports/ball03.png}
		\end{figure}
		Die Schule braucht höchstens 192 Bälle am Freitag um 13 Uhr.
	\subsection{Beispiel 04}
		\begin{figure}[H]
			\centering
			\includegraphics[scale=0.65]{exports/ball04.png}
		\end{figure}
		Die Schule braucht höchstens 30 Bälle am Montag um 10 Uhr.
	\subsection{Beispiel 05}
		\begin{figure}[H]
			\centering
			\includegraphics[scale=0.65]{exports/ball05.png}
		\end{figure}
		Die Schule braucht höchstens 25 Bälle am Donnerstag um 8 Uhr.
	\subsection{Beispiel 06}
		\begin{figure}[H]
			\centering
			\includegraphics[scale=0.65]{exports/ball06.png}
		\end{figure}
		Die Schule braucht höchstens 71 Bälle am Dienstag um 9 Uhr.
	\subsection{Beispiel 07}
		\begin{figure}[H]
			\centering
			\includegraphics[scale=0.65]{exports/ball07.png}
		\end{figure}
		Die Schule braucht höchstens 28 Bälle am Montag um 7 Uhr.


	\newpage
	\section{Quellcode}
	\subsection{Algorithmus}
		\begin{lstlisting}[language=Python,label={lst:code3}]
		class Unterricht:
			def __init__(self, Class, Day, Start, End, BallsNeeded):
				self.Class = Class
				self.Day = Day
				self.Start = int(Start)
				self.End = int(End)
				self.BallsNeeded = int(BallsNeeded)

			def __str__(self):
				return f"{self.BallsNeeded} zwischen {self.Start} und {self.End}"


		# Berechnet die Stunden in denen die Bälle benötigt werden
		def getHours(start, end):
			c = start
			hours = []
			while c <= end:
				hours.append(c)
				c = c + 1
			return hours

		def isSport(name):
			if name[0].isdigit():
				return True
			elif name.startswith("SP")  or name.startswith("sp"):
				return True
			else:
				return False

		def getData(file):
			with open('data/' + file) as f:
				lines = f.readlines()
				del lines[0] # die erste Zeile ist nicht relevant für diesen Zweck

				classes = []
				days = {"Montag": {}, "Dienstag" : {}, "Mittwoch": {}, "Donnerstag" : {},  "Freitag" : {}} # dictionary of Weekdays, with their corresponding Timestamps

				for line in lines:
					tempClass = line.split(" ")
					classes.append(Unterricht(tempClass[0],tempClass[1],tempClass[2],tempClass[3],tempClass[4].rstrip()))

				for item in classes:
					hours = getHours(item.Start, item.End)
					day_time_stamps = days[item.Day]
					for hour in hours:
						if isSport(item.Class):
							lastVal = day_time_stamps.get(hour)
							if lastVal is None: lastVal = 0
							day_time_stamps[hour] = lastVal + item.BallsNeeded
					days[item.Day] = day_time_stamps

				maxBalls = 0
				maxHour = 0
				maxDay = ""
				graphDict = {}

				# calculate maximum usage
				for dayKey, day in days.items():
					if len(day.values()) > 0:
						maxValue = max(day.values())
						if maxValue > maxBalls:
							maxBalls = maxValue
							maxHour = max(day, key=day.get)  # gets the hour corresponding to max value
							maxDay = dayKey  # stores the day name

				for day, hours in days.items():
					for hour, value in hours.items():
						graphDict[f"{day[0] + day[1]} {hour}"] = value
				return maxValue, maxHour, maxDay, graphDict
		\end{lstlisting}

	\subsection{GUI}
		\begin{lstlisting}[language=Python,label={lst:code4}]
			import os
			import Baellebad  as b
			import pyqtgraph as pg
			from PyQt6.QtWidgets import QLabel
			from PyQt6.QtWidgets import *
			import sys

			files = os.listdir("data")
			x = list()
			y = list()


			def getData(file):
				maxValue, maxHour, maxDay, timeStamps = b.getData(file)
				x_labels = list(timeStamps.keys())
				x = list(range(len(x_labels)))
				y = list(timeStamps.values())
				return maxValue, maxHour,maxDay, x_labels, x,y


			class Window(QMainWindow):
				def FileClicked(self,item):
					maxValue, maxHour, maxDay, x_labels, x, y = getData(item.text())
					self.bargraph.setOpts(x = x, height = y)
					self.resultText.setText("Die Schule braucht höchstens " + str(maxValue) + " Bälle am " + maxDay + " um " + str(maxHour) + " Uhr!")
					self.plot.getAxis('bottom').setTicks([list(zip(x, x_labels))])
					self.plot.removeItem(self.bargraph)
					self.plot.addItem(self.bargraph)

				def __init__(self):
					super().__init__()
					self.setWindowTitle("Bällebad")
					self.setGeometry(100, 100, 600, 500)
					self.UiComponents()
					self.show()

				def UiComponents(self):
					widget = QWidget()
					listWidget = QListWidget()
					listWidget.setMaximumWidth(300)
					listWidget.itemClicked.connect(self.FileClicked)
					for i in range(len(files)):
						listWidget.addItem(files[i])

					self.resultText = QLabel()

					self.plot = pg.plot()

					self.bargraph = pg.BarGraphItem(x = x, height = y, width = 0.6, brush ='b')
					self.plot.addItem(self.bargraph)
					layout = QGridLayout()
					widget.setLayout(layout)
					layout.addWidget(listWidget, 0, 0)
					layout.addWidget(self.plot, 0, 1)
					layout.addWidget(self.resultText, 1, 1)
					self.setCentralWidget(widget)


			App = QApplication(sys.argv)
			window = Window()
			sys.exit(App.exec())

		\end{lstlisting}

\end{document}

\documentclass[a4paper,10pt,ngerman]{scrartcl}
\usepackage{babel}
\usepackage[T1]{fontenc}
\usepackage[utf8x]{inputenc}
\usepackage[a4paper,margin=2.5cm,footskip=0.5cm]{geometry}

% Die nächsten vier Felder bitte anpassen:
\newcommand{\Aufgabe}{Aufgabe 2: Silbentrennung} % Aufgabennummer und Aufgabennamen angeben
\newcommand{\TeamId}{00015}                       % Team-ID aus dem PMS angeben
\newcommand{\TeamName}{TGG-Abi26}                 % Team-Namen angeben
\newcommand{\Namen}{Jannes Lehmann, Tino Brinker}           % Namen der Bearbeiter/-innen dieser Aufgabe angeben
\usepackage{float}
% Kopf- und Fußzeilen
\usepackage{scrlayer-scrpage, lastpage}
\setkomafont{pageheadfoot}{\large\textrm}
\lohead{\Aufgabe}
\rohead{Team-ID: \TeamId}
\cfoot*{\thepage{}/\pageref{LastPage}}

% Position des Titels
\usepackage{titling}
\setlength{\droptitle}{-1.0cm}

% Für mathematische Befehle und Symbole
\usepackage{amsmath}
\usepackage{amssymb}

% Für Bilder
\usepackage{graphicx}

% Für Algorithmen
\usepackage{algpseudocode}

% Für Quelltext
\usepackage{listings}
\usepackage{color}
\definecolor{mygreen}{rgb}{0,0.6,0}
\definecolor{mygray}{rgb}{0.5,0.5,0.5}
\definecolor{mymauve}{rgb}{0.58,0,0.82}
\lstset{
	keywordstyle=\color{blue},commentstyle=\color{mygreen},
	stringstyle=\color{mymauve},rulecolor=\color{black},
	basicstyle=\footnotesize\ttfamily,numberstyle=\tiny\color{mygray},
	captionpos=b, % sets the caption-position to bottom
	keepspaces=true, % keeps spaces in text
	numbers=left, numbersep=5pt, showspaces=false,showstringspaces=true,
	showtabs=false, stepnumber=2, tabsize=2, title=\lstname
}


% Diese beiden Pakete müssen zuletzt geladen werden
%\usepackage{hyperref} % Anklickbare Links im Dokument
\usepackage{cleveref}

% Daten für die Titelseite
\title{\textbf{\Huge\Aufgabe}}
\author{\LARGE Team-ID: \LARGE \TeamId \\\\
\LARGE Team-Name: \LARGE \TeamName \\\\
\LARGE Bearbeiter/-innen dieser Aufgabe: \\
\LARGE \Namen\\\\}
\date{\LARGE\today}

\begin{document}

	\maketitle
	\tableofcontents

	\vspace{0.5cm}
	\newpage

	\section{Lösungsidee}
		In der deutschen Sprache werden Silben anhand von zahlreichen Regeln getrennt.
		Für die Implementation dieser ist in der Regel höchstens ein Kontetfenster von 6 Zeichennötig.
		Manche Regel bedingen einander und werden daher hierarchisch behandelt.
		Andere Regeln stehen für sich alleine und werden immer angewandt.

	\section{Umsetzung}
		Die Lösung dieser Aufgabe wurde in drei separate Modulen interpretiert.
		\begin{enumerate}
			\item Silbentrennung.py zur Trennung der Silben
			\item GUI.py dient als praktische Anwendung unseres Algorithmus. Hier können eigene Texte eingegeben und getrennt werden.
			\item AutomaticTesting.py dient als Test des Algorithmus. Verschiedene Fälle werden durchgeführt und mit einem Ziel Wert verglichen.
		\end{enumerate}

		Im ersten Durchgang des Algorithmusses wird durch den Text iteriert und jeweils ein ganzes Wort in \lstinline|force_seperate(compound)| übergeben.
		Zuerst werden verschiedene Regeln, wie die Trennung vor einem "sch" angewand.
		\begin{lstlisting}[language=Python, caption={Trennung vor 'sch'},label={lst:code2}]
			schIndex = pass_compound.rfind("sch")
			if schIndex >= 1:
				schIndex = schIndex + 3
				if isConsonant(pass_compound[schIndex]):
					pass_compound = pass_compound[0:schIndex]
										+ " "
										+ pass_compound[schIndex: len(pass_compound)]
					result = " " + pass_compound
		\end{lstlisting}

		Daraufhin wird die Library german\_compound\_splitter verwendet und durch um zusammengesetzte Wörter in ihre Bestandteile zu trennen. \newline
		Sobald die erste Loop den Text bearbeitet hat, wird dieser noch ein zweites mal durch die Methode check\_for\_sillable(c), angepasst.
		In dieser Spiegelt c[1], das Zeichen nach dem eine Leertaste eingefügt werden soll. \newline

		Bsp für eine Regel:
		\begin{lstlisting}[language=Python, caption={Trennung nach einem Vokal, sollten keine 2 Konsonanten folgen},label={lst:code3}]
    		if isConsonant(c[1]) == False and ((isConsonant(c[2]) == False and isConsonant(c[3]))
				or (isConsonant(c[2]) and isConsonant(c[3]) == False)):
				state = True
		\end{lstlisting}
		Das setzen von "state" auf true sorgt jedoch nicht zwigend für die Trennung einer Silbe.
		Sollte eine Regel mit höherer Priorität, bzw. späterer Position im Code diese Variable auf False setzen, kommt es nicht zur Trennung.


	\section{Werkzeuge}
		Für das Programm wurden folgende drittanbieter Libraries verwendet:
		\begin{enumerate}
			\item PyQt6
			\item pyqtgraph
			\item german\_compound\_splitter
		\end{enumerate}
		Es wurden keine weiteren Hilfsmittel verwendet.

	\newpage
	\section{Beispiele}


	\newpage
	\section{Quellcode}
	\subsection{Algorithmus}
		\begin{lstlisting}[language=Python,label={lst:code4}]

		\end{lstlisting}

	\subsection{GUI}
		\begin{lstlisting}[language=Python,label={lst:code4}]


		\end{lstlisting}

\end{document}
